\documentclass[10pt]{article}

\usepackage[left=2cm, right=2cm, top=2cm, bottom=3cm]{geometry}

% Input encoding and typographical rules for English language
\usepackage[utf8]{inputenc}
\usepackage[english]{babel}
\usepackage[english]{isodate}

% tikz is used to draw images in this example, but you can
% also use \includegraphics{}.
\usepackage{tikz}
\usepackage{pgfplots}
\usepackage{circuitikz}
\usetikzlibrary{calc}

\usepackage{enumitem}
\usepackage{multicol}

\makeatletter
\renewcommand{\maketitle}{
{\Huge\textbf{\@title}} \\
\noindent\rule{\textwidth}{1pt} \\
\medskip
}
\makeatother

% Section header size and spacing
\makeatletter
\renewcommand{\section}{%
  \@startsection{section}{1}{0pt}{-3.5ex plus -1ex minus -.2ex}{2.3ex plus .2ex}{\Large\bfseries\sffamily}%
}
\makeatother

% No numbering for section titles
\setcounter{secnumdepth}{0}

% ============================================================================ %

% These define global texts that are used in headers and titles.
\title{mea-craft}
\author{Felipe Paiva Alencar}
\date{June 2023}
% \revision{Revision 1}

\begin{document}
\maketitle

\begin{multicols}{2}

\section{Features}

\begin{itemize}

\item \textbf{RISC-V Core:}
\begin{itemize}
\item DIY implementation of the RV32I instruction set architecture (ISA),
providing a flexible and customizable processing core.
\item Uses the AXI4-Lite memory interface enabling a seamless interface with
different memory devices.
\item Has support for interrupts allowing the implementation of event-driven
functionalities.
\end{itemize}

\item \textbf{Graphics:}
\begin{itemize}
\item The sprite architecture presented in class, has been enhanced to enable
dynamic changes to sprite contents, multiple texture scales, and memory sharing
between sprites.
\item Fully parametric, has support 80 sprites arranged in 5 clusters in the
default configuration.
\end{itemize}

\item \textbf{Memory:} 
\begin{itemize}
\item 4 kilobytes of ROM: Storing a small bootloader that also performs a quick
test of the ISA implementation.
\item 64 kilobytes of RAM: More than sufficient memory capacity to
handle game, texture, and world data.
\item 20.48 kilopixels of texture memory: arranged in 5 clusters of sprites.
\end{itemize}

\item \textbf{Peripherals:}
\begin{itemize}
\item General Purpose Input/Output (GPIO) that allows the interface of the
software with the buttons and switches. 
\item Universal asynchronous receiver-transmitter (UART) that provides a
communication channel.
\end{itemize}

\item \textbf{Build Tools:}
\begin{itemize}
\item The build and flashing process is efficiently automated by a well designed
Makefile, simplifying the compilation, texture packaging, world generation, and
linking tasks.
\item Support for bulding and flashing via a single command, saving valuable
development time and effort.
\end{itemize}

\end{itemize}


   

 



\end{multicols}


%     \item[\textbf{2.}] \textbf{Memory:}
%     \begin{itemize}
%         \item 
%         \item 
%     \end{itemize}
    
%     \item[\textbf{3.}] \textbf{Interrupt Support:} Enables efficient handling of interrupts, allowing you to
    
%     \item[\textbf{4.}] \textbf{2D GPU:}
%     \begin{itemize}
%         \item VGA Labs-based GPU: A simple 2D graphics processing unit based on VGA Labs, delivering a resolution of 20480 pixels in video memory, facilitating rich visual experiences in your game.
%         \item 80 Sprites: Supports up to 80 sprites, divided into 5 clusters, enhancing your game's ability to render dynamic and interactive graphics.
%     \end{itemize}
    
%     \item[\textbf{5.}] \textbf{Compiler:}
%     \begin{itemize}
%         \item Clang LLVM Compiler: Utilizes the modern Clang LLVM compiler, which provides efficient code generation and optimization capabilities, enabling faster and more optimized execution of your game code.
%     \end{itemize}
    
%     \item[\textbf{6.}] \textbf{Well-Made Makefile:} A carefully crafted Makefile automates the build process, streamlining the compilation, linking, and packaging of textures, world data, and other game assets, saving you valuable development time and effort.
% \end{itemize}

% Additional Specifications:
% \begin{itemize}
%     \item Power Supply: [Specify power supply requirements]
%     \item Clock Speed: [Specify clock speed or frequency]
%     \item I/O Interfaces: [Specify any additional input/output interfaces supported by the processor]
%     \item Operating Voltage: [Specify the required operating voltage range]
%     \item Debugging Support: [Specify any additional debugging capabilities or interfaces available]
% \end{itemize}

% \section{Applications}

% \begin{itemize}
% \item{Data sheets for electronics components}
% \item{Technical sales brochures}
% \item{Functional specifications}
% \end{itemize}

% \section{General Description}
% The \textbf{datasheet} document class makes it easy to write great looking
% data sheets using the LaTeX typesetting system. It follows the classic style used
% by most manufacturers of electronic components.

% You can download the document class from
% \href{https://github.com/PetteriAimonen/latex-datasheet-template/}{latex-datasheet-template}
% GitHub repository.
% The repository includes this example datasheet as \textbf{example.tex} and
% the document class as \textbf{datasheet.cls}.
% You can build the PDF document using command \texttt{latexmk -pdf}.

% % Switch to next column
% \vfill\break

% \begin{figure}[h]
%     \begin{circuitikz}[european]
%         \node[op amp] (amp1) {};
%         \node[op amp, below = 0.5cm, xscale = -1] (amp2) {};
%         \draw (amp1.out) |- (amp2.-);
%         \draw (amp2.-) ++(0, 0.3cm) node[circ]{} to +(2,0) node[above left]{5};
%         \draw (amp2.out) to (amp1.+);
%         \draw (amp1.+) ++(0, -0.3cm) node[circ]{} to +(-2,0) node[above right]{2};
%         \draw (amp1.-) to +(-2,0) node[above right]{1};
%         \draw (amp2.+) to +(2,0) node[above left]{4};
%         \draw (amp1.out) +(0,0.5cm) node (Vdd) {$\mathrm{V_{DD}}$};
%         \draw (Vdd.east) to +(1.5,0) node [above left]{6};
%         \draw (amp2.out) +(0,-0.5cm) node (Vss) {$\mathrm{V_{SS}}$};
%         \draw (Vss.west) to +(-1.6,0) node [above right]{3};
%         \draw ($(amp1.north west) + (-0.5,0.5)$) rectangle ($(amp2.south west) + (0.5,-0.5)$);
%     \end{circuitikz}
%     \caption{Pinout and internal circuit}
% \end{figure}

% \begin{figure}[h]
%     \begin{tikzpicture}
%         \sffamily
%         \begin{axis}[
%             width=7cm,
%             xlabel={Number of LaTeX packages},
%             ytick distance=20,
%             yticklabel={\pgfmathprintnumber{\tick}\%},
%             xmajorgrids, ymajorgrids]
%         \addplot[smooth,mark=*] plot coordinates {
%             (0,4)
%             (5,50)
%             (10,80)
%             (15,95)
%             (20,98)
%         };
%         \end{axis}
%     \end{tikzpicture}
%     \caption{Typical data sheet production efficiency}
% \end{figure}

% % For wide tables, a single column layout is better. It can be switched
% % page-by-page.
% \onecolumn

% \section{Electrical Specifications}
% All specifications are in $-40\degree C \leq T_A \leq 85\degree C$ unless otherwise noted.

% \begin{table}[h]
% \begin{threeparttable}
% \caption{Example Data Sheet Specifications}
% \begin{tabularx}{\textwidth}{l | c | c c c | c | X}
%     \thickhline
%     \textbf{Parameter} & \textbf{Symbol} & \textbf{Min.} & \textbf{Typ.} & \textbf{Max.} &
%     \textbf{Unit} & \textbf{Conditions} \\
%     \hline
%     Page width  & $p_w$ & 20.9 & 21.0 & 21.1 & cm & \multirow{2}{*}{Standard A4 paper} \\
%     Page height & $p_h$ & 29.6 & 29.7 & 29.8 & cm &  \\
%     \hline
%     Insulation voltage & $E_{max}$\tnote{1} & & 1 & & kV & \\
%     \thickhline
% \end{tabularx}
% \begin{tablenotes}
% \item[1]{Based on characterization data, not tested in production.}
% \end{tablenotes}
% \end{threeparttable}
% \end{table}

% \section{Absolute Maximum Ratings}

% \begin{table}[h]
% \caption{Absolute Maximum Ratings of Example Data Sheet}
% \begin{tabularx}{\textwidth}{l | X}
%     \thickhline
%     \textbf{Parameter} & \textbf{Rating} \hspace{5cm} \\
%     \hline
%     Daily exposure to LaTeX & 24 hours \\
%     \thickhline
% \end{tabularx}
% \end{table}

% \textbf{Note:} Stresses above those listed under Absolute Maximum Ratings can
% cause permanent damage to the device. This is a stress rating only. Functional
% operation of the device is not implied in any conditions above those indicated
% in the Electrical Specifications section.

\end{document}


